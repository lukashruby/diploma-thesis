% !TeX root = thesis.tex


\chap Implementation

In this chapter the main information about development of the project will be introduced. Firstly, used developer tools will be presented. The Android application and logging server structures will follow. At the end of the chapter will be a section about difficulties and problems and their solutions. 


\sec{Development environments}

\secc{Android SDK}

The Android SDK is a set of various tools that help developers to develop mobile applications for the Android platform. The tools are divided into two groups: SDK tools and platform tools. SDK tools are independent on target Android platform, platform tools are customizes to support concrete version of Android OS~\cite[androidsdkhelp].

To SDK tools belongs Virtual device tools (such as Android Virtual Device Manager or Android Emulator), Development tools (such as SQLite DB Explorer, SDK Manager or Translation editors), Debugging tools (Android Monitor, Android Debug Bridge, etc.), Build and Image tools. \rfc{jak je to s tema platform tools, patri tam adb nebo ne?}

\secc{Java Platform, Enterprise Edition}

Java Platform, Enterprise edition (Java EE) is the standard in community-driven enterprise software. Java EE is developed using the Java Community Process\urlnote{http://www.jcp.org/} with contributions from industry experts, commercial and open source organizations and many individual developers. Nowadays there are more than 20 compliant Java EE implementation~\cite[javaee].

\secc{Java Platform, Standard Edition}

\rfc{co je to vlastne java? tady uz to nebude tak lehky - existuje i neco jako SE Advanced edition, ktera je placena} \cite[javase]


\sec Development tools

\rfc{sth common about dev tools}


\secc Integrated development environment

For the purpose of this thesis was used the offical IDE for Android application development provided by Google - Android Studio. Android studio is based on IntelliJ IDEA, a commercial product
by the company JetBrains s.r.o.. Beside the standard set of functions from IntelliJ IDEA, Android Studio offers tailored-made extensions for Android application developers, therefore is the recommended option to start the development with. \rfc{cite source}

\secc Code version control system

Version control is a system that records changes to a file or set of files over time so that user can recall specific version later. Version control is mostly used in software development process to manage diffrent versions of source code. But \glref{VCS} is not limited only to plain text information. Graphics and other multimedia can be versioned as well~\cite[torvalds2010git].

There are three main approaches in version controlling:

\begitems
	* {\bf Local Version Control Systems} - all changes are stored only on the single computer used by user (e.g., simple file copying to other folder),
	* {\bf Centralized Version Control Systems} - versions are stored on the server and all clients access it from remote locations (e.g., Subversion, CVS),
	* {\bf Distributed Version Control Systems} - versions are stored on the server, but all cliets have local copy of the repository and are able to make changes independently (e.g., Git or Mercurial).
\enditems

\seccc Git

For the purpose of development the application, Git \glref{VCS} was chosen. Git was founded in 2005 by the Linux development comunnity (in particular Linus Torvalds, the creator of Linux) to be an alternative to the commercial product BitKeeper which was used as \glref{VCS} for development of the Linux kernel. The main objectives were~\cite[githistory]:

\begitems
	* Speed,
	* Simple design,
	* Strong support for non-linear development (thousants of parallel branches),
	* Fully distributed,
	* Ability to handle large projects like the Linux kernel.
\enditems

\seccc GitHub

GitHub\urlnote{https://github.com/} is a web-based Git repository hosting service. It was used for all developed software applications and tools in this thesis:

\begitems
	* Android application for smartphone\urlnote{https://github.com/eclubprague/CarDashboardPhone}
	* Shared core for Android platform\urlnote{https://github.com/eclubprague/CarDashboardCore}
	* Logging server\rfc{add logging server to git!}
	* OBD Bluetooth Emulation tool\urlnote{https://github.com/lukashruby/obd-mocked-bluetooth-dongle}
	* Thesis \TeX~source\urlnote{https://github.com/lukashruby/diploma-thesis}
\enditems

\secc Continuous integration service

Travis CI \rfc{todo}

\secc Build automation system

Graddle \rfc{todo}

\sec{Third party libraries}

During the development of the application and logging server were used also the third party libraries. The following list contains name of each library (or dependency) and reason why it was used in the project:

\begitems
	*https://github.com/bauerca/drag-sort-listview
	
	*com.github.pires:obd-java-api:1.0-RC6
	*com.squareup.okhttp:okhttp:2.5.0
	
	*commons-fileupload
	*org.mongodb.mongodb-driver
	
\enditems


\secc Mobile application

\rfc{todo}

\secc Logging service

\rfc{todo}

\sec Mobile Application Implementation

Mobile application is into two parts. The first part is platform dependent (smartphone in this case) and will be  described in detail in section~\ref[cdphone]. The second part, so-called {\bf Core}, is platform independent and can be ported to other devices which use Android OS (tablet, smart tv, etc.) In~\cite[blaha] is this part used in tablet version of this application. {\bf Core} was developed in cooperation with the author of \cite[blaha]. Details of the contribution is available in Core's Github repository\urlnote{https://github.com/eclubprague/CarDashboardCore}. From this part of project, only the important parts will be described in~\ref[core].

\label[cdphone]
\secc User interface for smartphones

Design of the user interface was the most imporant part of chapter \ref[design] a it is key problem of whole project. As this application is aimed to be used in car while driving, bad implementation of the concept can cause many difficulties for drivers. Therefore was very important to be especially precious in the development and this phase took majority of time.

Beside of the standard mode of application, when phone is placed firmly in holder and a driver uses single finger to change views, there is also the configuration mode. In the configuration mode are options to enable or disable {\em Bluetooth support} (for connection to OBD II Dongle) and to configure {\em Bluetooth devices}. Also the settings of {\em Logging features} is present in this mode. Last but not least function is {\em Module management}.
Each part of the application will be described in dedicated subsections.

\seccc ScreenSlideActivity \& ScreenSlidePageFragment

\seccc DnDActivity \& DnDFragment

\label[core]
\secc Shared Core Library

%\rfc{module adding - simplicity, updating, persistence, obdservice, logging}

\secc Results 

\rfc{maybe different name}

pictures

\sec Server implementation

\sec Reference Hardware

\secc Mobile devices

\seccc LG Nexus 5

\secc OBD II devices

\seccc MINI ELM327 bluetooth OBD2 V1.5 Bluetooth dongle

\seccc OBD PC Emulator
 






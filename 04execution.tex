% !TeX root = thesis.tex


\chap Implementation

In this chapter the main information about development of the project will be introduced. Firstly, used developer tools will be presented. The Android application and logging server structures will follow. At the end of the chapter will be a section about difficulties and problems and their solutions. 


\sec{Development environments}

\secc{Android SDK}

The Android SDK is a set of various tools that help developers to develop mobile applications for the Android platform. The tools are divided into two groups: SDK tools and platform tools. SDK tools are independent on target Android platform, platform tools are customizes to support concrete version of Android OS~\cite[androidsdkhelp].

The SDK tools consist of  Virtual device tools (such as Android Virtual Device Manager or Android Emulator), Development tools (such as SQLite DB Explorer, SDK Manager or Translation editors), Debugging tools (Android Monitor, Android Debug Bridge, etc.), Build and Image tools. 

\secc{Java Platform, Standard Edition}

Java SE is widely used platform for development and deployment of portable code for desktop and server environments. It uses the object-oriented programming language called Java. Java SE defines general purpose APIs, such as Java APIs for Java Class Library. It also includes the Java Language Specification and Java Virtual Machine Specification~\cite[javase]. There are many implementation of Java SE, most known are Oracles's Java Development Kit (JDK) and OpenJDK. 

\seccc{Oracle v. Google}

Android architecture has been using a Java Class Library derived from the Apache Harmony\urlnote{http://harmony.apache.org/} project supported by IBM. After Sun Microsystems was bought by Oracles in 2011, support of Apache Harmony ended (IBM joined OpenJDK) and Oracle sued Google for breaching the copyright law (by using Java APIs in Android)\urlnote{https://www.eff.org/cases/oracle-v-google}. 

With the new version of Android OS -- Android L -- Google is switching to OpenJDK to bring support of new features implemented in last versions of Java (e.g., Lambda operators)\urlnote{http://www.theregister.co.uk/2015/12/30/android_openjdk/}.


\secc{Java Platform, Enterprise Edition}

Java Platform, Enterprise edition (Java EE) is the standardized platform in community-driven enterprise software which extends Java Standard Edition. Java EE is developed using the Java Community Process\urlnote{http://www.jcp.org/} with contributions from industry experts, commercial and open source organizations and many individual developers. Nowadays there are more than 20 compliant Java EE implementation~\cite[javaee].



\sec Development tools

Development tools allow developers to implement, test and debug their code. The whole development process is possible even without these tools, but it would take more time to gain the same results. During the development of the mobile application and other software tools, different development tools were used.

\secc Integrated development environment

For the purpose of this thesis the official IDE for Android application development provided by Google was used. Android studio is based on IntelliJ IDEA, a commercial product by the company JetBrains s.r.o. Beside the standard set of functions from IntelliJ IDEA, Android Studio offers tailored-made extensions for Android application developers, therefore it is the recommended option to start the development with~\cite[androidstudio].

\secc Code version control system

Version control is a system that records changes to a file or set of files over time so that user can recall specific version later. Version control is mostly used in software development process to manage different versions of source code. But \glref{VCS} is not limited only to plain text information. Graphics and other multimedia can be versioned as well~\cite[torvalds2010git].

There are three main approaches in version controlling:

\begitems
	* {\bf Local Version Control Systems} - all changes are stored only on the single computer used by user (e.g., simple file copying to other folder),
	* {\bf Centralized Version Control Systems} - versions are stored on the server and all clients access it from remote locations (e.g., Subversion, CVS),
	* {\bf Distributed Version Control Systems} - versions are stored on the server, but all clients have local copy of the repository and are able to make changes independently (e.g., Git or Mercurial).
\enditems

\seccc Git

For the purpose of development the application, Git \glref{VCS} was chosen. Git was founded in 2005 by the Linux development community (in particular Linus Torvalds, the creator of Linux) to be an alternative to the commercial product BitKeeper which was used as \glref{VCS} for development of the Linux kernel. The main objectives were~\cite[githistory]:

\begitems
	* Speed,
	* Simple design,
	* Strong support for non-linear development (thousands of parallel branches),
	* Fully distributed,
	* Ability to handle large projects like the Linux kernel.
\enditems

\seccc GitHub

GitHub\urlnote{https://github.com/} is a web-based Git repository hosting service. It was used for all the developed software applications and tools in this thesis:

\begitems
	* Android application for smartphone\urlnote{https://github.com/eclubprague/CarDashboardPhone}
	* Shared core for Android platform\urlnote{https://github.com/eclubprague/CarDashboardCore}
	* Logging server\rfc{add logging server to git!}
	* OBD Bluetooth Emulation tool\urlnote{https://github.com/lukashruby/obd-mocked-bluetooth-dongle}
	* Thesis \TeX~source\urlnote{https://github.com/lukashruby/diploma-thesis}
\enditems

\secc Continuous Integration Service

Continous Integration (CI) is a software development practice where developers are integrating their work frequently. It is used in development teams of all sizes. Usually the code is automatically verified, tested, and built to prove that changes do not have any hidden side effects. It is very important to set up whole build process properly. \glref{CI} is often connected to \glref{VCS}. For the purpose of development the mobile application, Travis CI\urlnote{https://travis-ci.org/} was used.

\secc Build Automation System

As mentioned in previous section, automated builds are important in \glref{CI} process. But well-prepared Build automation system can save a considerable amount of time. Automated builds are compiling source code, resolving dependencies,  packaging binary code, and running automated tests. All these task are executed directly from developer's computer or remotely. Most known Build automation systems are: GNU Make\urlnote{https://www.gnu.org/software/make/}, Apache Ant\urlnote{http://ant.apache.org/}, Apache Maven\urlnote{https://maven.apache.org/} and Graddle\urlnote{http://gradle.org/}.

\sec{Third party libraries}

During the development of the application and logging server the third party libraries were used. The following sections contain names of the libraries (or dependencies) and reason why they were used in the project.

\secc Mobile application

In mobile application third party libraries were used mainly for the purposes of communication (with a car \glref{ECU} or with server). One library also implements drag and drop feature, which is described in \ref[cdphone].

\begitems
	
	*OBD-II Java API\urlnote{https://github.com/pires/obd-java-api} - support for OBDII PIDs in serial communication,
	*OkHttp\urlnote{https://github.com/square/okhttp} - an HTTP client for Android applications, used for communication with server,
	*DragSortListView\urlnote{https://github.com/bauerca/drag-sort-listview} - implementation of drag and drop used for module management.
\enditems

\secc Logging service

As logging server receives files from mobile clients and saves the content to the database, third party libraries were used to simplify development of these features.

\begitems
	*Commons FileUpload\urlnote{https://commons.apache.org/proper/commons-fileupload/} - robust, high-performance, file upload capability for servlets and web applications,
	*MongoDB Java Driver\urlnote{https://mongodb.github.io/mongo-java-driver/} - provides synchronous and asynchronous interaction with Mongo databases.
\enditems

\sec Mobile Application Implementation

Mobile application is divided into two parts. The first part is platform dependent (smartphone in this case) and will be  described in detail in section~\ref[cdphone]. The second part, so-called {\bf Core}, is platform independent and can be ported to other devices which use Android OS (tablet, smart tv, etc.) In~\cite[blaha] this part is used in tablet version of this application. {\bf Core} was developed in cooperation with the author of \cite[blaha]. Details of the contribution is available in Core's Github repository\urlnote{https://github.com/eclubprague/CarDashboardCore}. From this part of project, only the important parts will be described in~\ref[core].

\label[cdphone]
\secc User interface for smartphones

Design of the user interface was the most important part of chapter \ref[design] \rfc{add chapter} and it is a key problem of the whole project. As this application is aimed to be used in a car while driving, bad implementation of the concept can cause many difficulties for drivers. Therefore it was very important to be especially precious in the development and this phase took the majority of time.

Standard mode of application consists of various {\bf Modules}. Mobile phone version displays only one module at a time (tablet version \cite[blaha] contains more modules at the screen). These modules are ordered in a tree structure -- a single module is displaying some information or contains more modules. 

Beside of the standard mode of application, when phone is placed firmly in holder and a driver uses single finger to change views, there is also the configuration mode. In the configuration mode there are options to enable or disable {\em Bluetooth support} (for connection to OBD II Dongle) and to configure {\em Bluetooth devices}. Also the settings of {\em Logging features} is present in this mode. Last but not least function is {\em Module management}.
Each part of the application will be described in dedicated subsections.

\seccc ScreenSlideActivity \& ScreenSlidePageFragment

Prevailing part of user interface is the implementation of vertical scrolling list of modules in {\tt ScreenSlideActivity}. This list is presented as infinite menu (after last item in list comes the first one in loop). Android provides support only for horizontal scrolling list, so this implementation was done from scratch. {\tt ScreenSlidePageFragment} is then the content of single Module. It consists of a Module name, a Module icon, a Module value and a Module units.


\seccc DnDActivity \& DnDFragment 

DnD means in this case abbreviation for Drag and Drop. This feature is used in configuration part, where user can change order of modules, add new modules and delete modules with simple swiping gestures. This approach makes the configuration process easy and straightforward. 

\label[core]
\secc Shared Core Library

\rfc{module adding - simplicity, updating, persistence, obdservice, logging}

\secc Final application

\midinsert
\line{\hsize=.5\hsize \vtop{%
      \clabel[module.speed]{Module displays single information.}
      \picw=5cm \cinspic resources/images/screenshots/speed.png
      \caption/f Module displaying single information.
   \par}\vtop{%
      \clabel[module.folder]{Module contains another set of child modules - folder.}
      \picw=5cm \cinspic resources/images/screenshots/folder.png
      \caption/f Module contains another set of child modules - folder.
   \par}}
\endinsert

\midinsert
\line{\hsize=.5\hsize \vtop{%
      \clabel[module.swipe]{Swiping between different modules.}
      \picw=5cm \cinspic resources/images/screenshots/swipe.png
      \caption/f Swiping between different modules.
   \par}\vtop{%
      \clabel[module.btdev]{Selection of Bluetooth device to connect to OBD dongle.}
      \picw=5cm \cinspic resources/images/screenshots/btdev_selection.png
      \caption/f Selection of Bluetooth device to connect to OBD dongle.
   \par}}
\endinsert

\midinsert
\line{\hsize=.5\hsize \vtop{%
      \clabel[module.change]{Changing position of module using Drag and drop.}
      \picw=5cm \cinspic resources/images/screenshots/dnd_change_position.png
      \caption/f Changing position of module using Drag and drop.
   \par}\vtop{%
      \clabel[module.remove]{Removing module using simple swipe gesture.}
      \picw=5cm \cinspic resources/images/screenshots/dnd_remove.png
      \caption/f Removing module using simple swipe gesture.
   \par}}
\endinsert


\sec Server implementation

An important part of the assignment of this thesis is implementation of a logging server. Purpose of the server is to gather data from devices with installed mobile application and provide these data to further exploration. 
Server was implemented as HTTP service in Java EE using Servlet API. It can be deployed to various application servers such as Apache Tomcat or Oracle Glassfish. Oracle Glassfish\urlnote{http://www.oracle.com/us/products/middleware/cloud-app-foundation/glassfish-server/overview/index.html} was used for deployment in this thesis. 

\secc Communication with server

Mobile application is logging all OBD data to the file stored in local storage. As these logs may be very data intensive, it is not reasonable to upload them live. Therefore when a wi-fi connection is available, user can initiate the upload of all files.  






% !TeX root = thesis.tex


\chap Execution

In this chapter the main information about development of the project will be introduced. Firstly, used developer tools will be presented. The Android application and logging server structures will follow. At the end of the chapter will be a section about difficulties and problems and their solutions. 


\sec{Development environments}

\secc{Android SDK}

\secc{Java Platform, Enterprise Edition}

\secc{Java Platform, Standard Edition}

\sec Development tools

\rfc{sth common about dev tools}


\secc Integrated development environment

For the purpose of this thesis was used the offical IDE for Android application development provided by Google - Android Studio. Android studio is based on IntelliJ IDEA, a commercial product
by the company JetBrains s.r.o.. Beside the standard set of functions from IntelliJ IDEA, Android Studio offers tailored-made extensions for Android application developers, therefore is the recommended option to start the development with. \rfc{cite source}

\secc Code version control system

Git \& github

\secc Continuous integration service

Travis CI

\secc Build automation system

Graddle

\sec{Third party libraries}

\secc Mobile application

\secc Logging service

\sec Reference Hardware

\secc Mobile devices

\seccc LG Nexus 5

\seccc Samsung Galaxy S3 Neo

\seccc Samsung Galaxy S5

\secc OBD II devices

\seccc ELM 321 Bluetooth dongle \rfc{corrent name}

\seccc OBD PC Emulator
 






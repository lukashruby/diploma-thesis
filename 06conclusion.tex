% !TeX root = thesis.tex

\chap Conclusion

In the last chapter of this thesis, the assignment competition will be analyzed. For every task, the relevant section will be referenced. After this analysis will come the section that summarizes the general work on this thesis and its consequences. The last section of this chapter (and thesis) is dedicated to future plans.


\sec Assignment completion

\secc Completing the assignment tasks

\seccc Study the current trends in controlling car infotainment systems

Current trends in design and control of infotainment systems are summarized in section \ref[1.types]. Historical evolution and current state are compared. The complexity of currently used devices in in section \ref[1.examples].

\seccc Explore existing mobile applications for in-car use

The most significant producers of mobile applications for in-car use are mentioned in section \ref[2.incarapps]. In the chapter \ref[analysis] is overall described current market, the shipments of mobile devices, and a variety of available mobile operating systems.

\seccc Study communication with car ECU (CAN, OBD, etc.)

Data and communication inside a vehicle is described in section \ref[2.data]. Different types of in-car buses are listed in section \ref[2.buses]. Moreover in section \ref[2.othercomm] is also the analysis of communication with the outer world. 

\seccc Design and implement mobile application for in-car user

This task is the topic of chapter \ref[design] and \ref[implementation]. In chapter \ref[design] the design process in thoroughly described including the general guidelines. In chapter \ref[implementation] the implementation phase is discussed. It contains description of used environment, software tools, third party libraries, and the mobile application implementation.  

\seccc Implement logging server

Process of design and development of the logging server for purposes of storing driving data is described in section \ref[2.cloud], where the different concepts of cloud computing are introduces, then in section \ref[3.server] the logging server architecture is proposed and finally in section \ref[4.server] the implementation is described.

\seccc Integrate mobile application with logging server

Basics of possible ways of communication are listed in section \ref[2.othercomm] and particular method used in the development is described in section \ref[4.comm].

\seccc Compare your solution with existing solutions

Comparison with the biggest competitor, application Torque, is presented in section \ref[5.ab]. This comparison was done using the A/B tests on a car simulator with users. 

\seccc Evaluate the application using usability testing

Two types of usability tests were performed. A/B tests for the purpose of compare the application CarDashboard with the competitor. Second type of tests, Lane Change Test, was executed on a car simulator as well. Progress and results of usability testing are available in section \ref[5.lct], \ref[5.ab], and in appendix \ref[tests_results].


\sec Summary

The goal of this thesis was to examine a new approach in development of in-car mobile applications. This goal was successfully reached. Along the way of solving the particular tasks, many other tools were developed to support the main goal, such as shared library CarDashboard Core, OBD II emulation Java tool and Logging server. These tools can be used even without the main mobile application CarDashboard.

My first contact with the design an in-car user interface and the measurement of cognitive load was during my Erasmus studies at Johannes Kepler University in Linz, Austria. I took the course Principles of Interaction led by Prof. Dr. Andreas Riener, leading expert in this discipline. At those time I did not fully understand all details of this field. Still, this topic was very interesting and after my return to the Czech Republic I decided to carry on under supervision of Dr. Jan Šedivý.

\midinsert \clabel[phone_range]{Final CarDashboard mobile application with the day mode color scheme}
\picw=\hsize \cinspic resources/images/phone_range.png 
\caption/f Final CarDashboard mobile application with the day mode color scheme
\endinsert

Although the mobile application CarDashboard performed very well in usability test in a car simulator, in-car environment is not the only way where the application can be utilized. After minor changes, it can be used as an universal control panel for the concept of smart building. 

During the work on this thesis and relevant project, I have learned various of new skills. First of all, I have got the perfect overview of the Android platform and how to make applications for it. Secondly, I have explored different ways of integration of software systems in local area and wide area networks. In third place, I have started to follow the current trends in automotive user interfaces and discovered that this field is much bigger then I had expected. And lastly, I have had to defend my idea and solutions, discuss them and improve them on numerous meeting with my supervisor and my colleges. Another interesting experience was setting up a cooperation with people from Faculty of Transport Sciences of CTU in Prague, planning the way of usability test execution and solving other challenging problems connected with usability testing. We are on a good way to continue with this so-far successful cooperation.

In the end, I would like to notice it was a great experience working on this topic and I would like to stay in touch with this field in further phases of my life.

\midinsert \clabel[phone_iot]{Final CarDashboard mobile modified for use in smart buildings}
\picw=10cm \cinspic resources/images/screenshotIoT.png 
\caption/f Final CarDashboard mobile modified for use in smart buildings
\endinsert


\sec Future work

Even though a large amount of work was done, there are still some tasks left. The most important task that will come in next weeks after defending this thesis is the publishing of the mobile application CarDashboard. The new name will be invented and a new graphic identity of whole application (logotype, colors, etc.) will be introduced. Based on the results from usability tests, we are expecting the application to be successful. Application will be published on the official application store by Google -- Google Play Store and every owner of an Android based smartphone will be able to download it.

After bringing the application to the general public, many new issues with different device combinations will certainly appear. But that is the destiny of every mobile application. We will do our best to provide our users the best possible experience from using the applications.






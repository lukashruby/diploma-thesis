% !TeX root = thesis.tex

\chap Analysis

\rfc{basic info about current trends, who are the people using smartphone while driving}

\sec In-car devices and controls

\rfc{small analysis on cars interiors then and now, increasing number of control}

\sec In-car mobile applications
 
 \rfc{main vendors, small apps on different stores}

\secc Android Auto

\secc Apple CarPlay

\secc Other vendors

\sec Data \& comumunication

\rfc{busses in car, OBD, CANBUS, FLEXRAY, BlueTooth, Internet connection}

\sec Android mobile platform

Android is a software stack for mobile devices that includes an operating system, middleware and key applications. The various components of Android are designed as a stack, with the ‘Applications’ forming the top layer of the stack, while the Linux kernel forms the lowest layer. Android ships with a set of core applications including an e-mail client, SMS program, calendar, maps, browser, contacts, and other features. All applications are written using the Java programming language.

Developers have full access to the same framework APIs used by the core applications. The application architecture is designed to simplify the reuse of components; the capabilities of any application can be published and then be made use of by any other application (subject to security constraints enforced by the framework). This same mechanism allows components to be replaced by the user.

Android includes a set of core libraries that provides most of the functionality available in the core libraries of the Java programming language. Every Android  aplication runs in its own process, with its own instance of the Dalvik virtual machine. Dalvik has been written so that a device can run multiple VMs efficiently. The Dalvik VM executes files in the Dalvik Executable (.dex) format, which is optimised for minimal memory footprint. The VM is register-based, and runs classes compiled by a Java language compiler that has been transformed into the .dex format by the included dx tool. The Dalvik VM relies on the Linux kernel for underlying functionality such as threading and low-level memory management. \cite[developers2011android]

\sec Cloud computing


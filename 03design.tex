% !TeX root = thesis.tex

\chap Design

\sec Requirements

\sec Android Design Guidelines

In section \ref[android5] was mentioned that with the realease of Android 5 Lollipop, Google provided guidelines for application developers to design user interface in proper way. These guidelines are called Google Design Guidelines~\urlnote{https://www.google.com/design/spec/material-design/introduction.html} and defines various design rules and parameters. The most important rule is that the display of the device shoud be taken as a real-world 3D environment. The whole concept is called Material Design~\urlnote{https://www.google.com/design/spec/what-is-material/environment.html}.


\sec Infotainment architecture

\secc Portrait vs. Landscape mode

The smartphone changes the user interface based on how the user holds it. The screen in the car can also be in the landscape or portrait mode. The landscape positioning is used by most of the cars on the market. The HP is showing the icons of the key applications and the user is choosing with the pointing device. The icons are fixed or move towards the central screen position. The horizontal pointing is used to move to both sides. The landscape position is on the other side not good for displaying list of names. It is easy to prove this statement. When we search a person in a contact application we can see many more lines in the portrait mode. 

The HP can be designed for both directions the same way, simply moving the focus left to right or up and down. The radio application, where probably the number one action is changing the station, probably the better format is portrait, since larger number of stations can be displayed. The media application is not that clear the forward, fast forward or backward adn stop buttons are traditionally arranged horizontally. On the other hand the selection of an artist, song, album, genre etc. will be better in the portrait mode. The navigation map uses in most cases the landscape screen, but again to select the destination is again a selection from a list where the portrait positioning might give an advantage. As we have mentioned above for the phone is probably better the portrait mode. The last key application is the setting of the car parameters. In this case we deal with large number of params, where a portrait mode may have an advantage.

It is not easy to lean on either side in selecting the screen positioning. However strictly stickin to one direction of moving between elements on the screen, it is going between icons and selecting from a list of text may have an advantage. If we would move only horizontally or vertically we will save on the pointing devices, the UI may be more intuitive easier to learn. Imagine an portrait arrangement, one can move around icons and change lines in lists of names only up and down. This can be easily controlled by two buttons up and down. 

The portrait mode may be also easier to move in between the speedometer and tachometer. Even if the portrait mode is rare it may have some advantages in the amount of information on the screen. However the final selection of the screen positioning is a usability study problem. 

\secc Usage modes 

\rfc{different name, maybe use cases}

\seccc Using while driving

\seccc Setting up

\sec GUI

\secc Low fidelity prototype

\secc High fidelity prototype

\sec Comunnication 

\sec Modularity

\sec Logging server architecture

\sec Other tools

\secc OBD II Emulator



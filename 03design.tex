% !TeX root = thesis.tex

\chap Design

\sec Requirements

\begitems
	* Simplicity of usage
	* Low cognitive load
	* Extensibility of functions
	* Logging feature
\enditems

\sec Android Design Guidelines

In section \ref[android5] was mentioned that with the release of Android 5 Lollipop, Google provided guidelines for application developers to design user interface in proper way. These guidelines are called Google Design Guidelines~\urlnote{https://www.google.com/design/spec/material-design/introduction.html} and defines various design rules and parameters. The most important rule is that the display of the device should be taken as a real-world 3D environment. The whole concept is called Material Design~\urlnote{https://www.google.com/design/spec/what-is-material/environment.html}.

\sec Comunnication 

As the application supports to communication with a car's \glref{ECU}, connects to the internet and uploads data logs, various types of communication are taken in account.

\secc Bluetooth

Support of Bluetooth technology is crucial for the final application. Bluetooth is used for connection to ELM327 Bluetooth Dongle which provides communication with \glref{OBD} interface. Dongles based on ELM327 usually supports USB, RM-232, WiFi or Bluetooth connections. In this case the Bluetooth connection was chosen because of simplicity of usage, commercial availability of the dongles and low power consumption\urlnote{http://science.opposingviews.com/bluetooth-vs-wifi-power-consumption-17630.html}.

\secc Wi-fi

\secc Mobile telecommunication

Category of mobile telecommunication in this case includes the possibilities of connection to the internet when wi-fi connection is not available. To this category belong wireless network technologies provided by mobile carriers (3G, 4G). This type of internet connectivity is not supposed to be used for uploading trip logs due to limited data plans issue. The typical use is to control other devices via HTTP request initiated from the smartphone (e.g., opening garage doors or entrance gates). The application is, however, usable without the data plan.

\sec Modularity


\sec Infotainment user interface architecture 

\rfc{maybe different title}
Based on the previous research (led by dr. Sedivy, supervisor of this thesis) the simple user interface structure was proposed. The concept of on-board computers, where the amount of information displayed at a time is limited, was adopted. The general overview of this concept is at figure~\ref[skoda_ui]. 

\midinsert \clabel[skoda_ui]{Limited information shown at display}
\picw=10cm \cinspic resources/images/skodaUI.pdf 
\caption/f Limited information shown at display (credits: dr. Sedivy)
\endinsert

\secc Portrait vs. Landscape mode

\rfc{tohle je nejakej copy paste -> prepsat}
The smartphone changes the user interface based on how the user holds it. The screen in the car can also be in the landscape or portrait mode. The landscape positioning is used by most of the cars on the market. The HP is showing the icons of the key applications and the user is choosing with the pointing device. The icons are fixed or move towards the central screen position. The horizontal pointing is used to move to both sides. The landscape position is on the other side not good for displaying list of names. It is easy to prove this statement. When we search a person in a contact application we can see many more lines in the portrait mode. 

The HP can be designed for both directions the same way, simply moving the focus left to right or up and down. For the radio application, where probably the number one action is changing the station, probably the better format is portrait, since larger number of stations can be displayed. The media application is not that clear: the forward, fast forward or backward and stop buttons are traditionally arranged horizontally. On the other hand the selection of an artist, song, album, genre etc. will be better in the portrait mode. The navigation map uses in most cases the landscape screen, but again to select the destination is a selection from a list where the portrait positioning might give an advantage. As we have mentioned above, for the phone is probably better the portrait mode. The last key application is the setting of the car parameters. In this case we deal with large number of parameters, where a portrait mode may have an advantage.

It is not easy to lean on either side in selecting the screen positioning. However strictly stickin to one direction of moving between elements on the screen, it is going between icons and selecting from a list of text may have an advantage. If we would move only horizontally or vertically we will save on the pointing devices, the UI may be more intuitive and easier to learn. Imagine a portrait arrangement, one can move around icons and change lines in lists of names only up and down. This can be easily controlled by two buttons up and down. 

The portrait mode may be also easier to move in between the speedometer and tachometer. Even if the portrait mode is rare it may have some advantages in the amount of information on the screen. However the final selection of the screen positioning is a usability study problem. 

\secc Use cases

There are two most important use cases for which is this application designed. \rfc{extend}

\seccc Using while driving

\seccc Setting up


\sec Graphical User Interface

User interface is one of the most important part of this thesis. \rfc{dodelat}

\cite[poi_design]

\secc Low fidelity prototype

\midinsert
\line{\hsize=.5\hsize \vtop{%
      \clabel[balsamiq_v1]{Low Fidelity mock up version 1}
	\picw=5cm \cinspic resources/images/mockups/v1.png
      \caption/f Low Fidelity mock up version~1
   \par}\vtop{%
      \clabel[balsamiq_v2]{Low Fidelity mock up version 2}
      \picw=5cm \cinspic resources/images/mockups/v2.png
      \caption/f Low Fidelity mock up version~2
   \par}}
\endinsert


Low fidelity prototypes are generally prototypes of limited functionality and interaction. They are constructed to depict concepts, design alternatives, and screen layouts. They are constructed quickly and provide limited or no functionality. These prototypes are created to demonstrate the general look, they are not intended to show in detail how the application operates. They should communicate, educate, and inform, not train, test, or serve as a basis from which to code~\cite[Rudd:1996:LVH:223500.223514].

For the purpose of creating low fidelity prototype of the application, the tool Balsamiq Mock-ups\urlnote{https://balsamiq.com/} was used. It is a rapid wireframing tool that reproduces the experience of sketching on a whiteboard, but using a computer.

The two step approach was applied. In the first step the ideas from previous chapters were used and the first version of lo-fi prototype was sketched. The result is at \rfc{preposition?} figure~\ref[balsamiq_v1]. After discussion and basic testing the mock-up was improved and the second version was designed (figure~\ref[balsamiq_v2]).

\secc High fidelity prototype

Unlike low fidelity prototypes, high fidelity prototypes are fully functional and interactive. They address the issues of navigation and flow and of matching the design and user models of a system. High fidelity prototypes are used for exploration and tests. Users can operate them the same way as the final product and they can make informed recommendations about how to improve the user interface~\cite[Rudd:1996:LVH:223500.223514].

High fidelity prototype of the application developed in this thesis was made in the target application's platform and continuously improved.



\sec Logging server architecture

\sec Other tools

\secc OBD II Emulator


